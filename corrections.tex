% -*-mode: japanese-latex; coding: utf-8-unix; -*-
%!TEX root = corrections.tex
\RequirePackage{plautopatch}
\documentclass[dvipdfmx,uplatex]{jsarticle}
\usepackage{amsmath}
\begin{document}
\title{湯川諭「統計力学」(日本評論社、2021)正誤表}
\author{湯川諭}
\date{\today}
\maketitle
\begin{itemize}
\item
p. 136 (8.9)式。
右辺の$\phi(\boldsymbol{q})$を$\phi_{k}(\boldsymbol{q})$に訂正。
\item
p. 139 (8.17)式。このままでも間違いではないですが、
ここまでのテキスト中の表記と合わせて$\Psi_{2,1}(\boldsymbol{q}_{A}, \boldsymbol{q}_{B})$などを
$\Psi_{1,2}(\boldsymbol{q}_{B}, \boldsymbol{q}_{A})$と書いた方が適切だったかもしれません。
\item
p. 168 最後の行「温度を一定にして体積が増加」を「温度を一定にして密度が増加」に訂正。
\item
p. 174 (10.46)式。
正しくは
\[
 \dfrac{C_V}{N k_\mathrm{B}}\simeq
 \dfrac{5}{2} \dfrac{\Gamma(5/2)}{\Gamma(3/2)} -
 \dfrac{9}{2} \dfrac{\Gamma(3/2)}{\Gamma(1/2)}
\]
\item
p. 199 12.1節「相と相転移」2行目「非凝集相」を「非凝縮相」に変更。10章の用語と統一する。8行目も同様に「凝集」を「凝縮」に変更。16行目「凝集する」から「凝縮する」に変更。
\item
p. 203 (12.6)式の上の行「臨界圧力」を「臨界密度」に訂正。
\item
p. 223 (12.88)式 「$\Delta Q$」の前に「$\beta$」を追加。
\item
p. 238 本文5行目「負号が逆」を「符号が逆」に訂正。
\item 
p. 238 問題10.1略解 「非凝集相」を「非凝縮相」に変更。本文の用語と統一する。
\item
p. 238 問題10.1略解 1行目から2行目にかけて「凝集相での一粒子あたりの体積」を「基底状態に凝縮した粒子の一粒子あたりの体積」に訂正。
3行目の「凝集相での一粒子あたりの体積」も同様に訂正。
\item
p. 238 問題10.2 略解2行目「非凝集相」を「非凝縮相」に変更。本文の用語と統一する。
\item
p. 239 略解2行目「臨界点」を「転移点」に変更。本文の用語と統一する。
\item
p. 241 本文7行目、問題12.2の解答において、 二箇所の$\langle \dots \rangle_{\mathrm{eq}}$の後ろに$Z$を掛け算で追加。
\end{itemize}
\end{document}
